\chapter{Sumatra PDF Viewer}  \label{kap:sumatra}

Statt den Adobe Reader zu verwenden, bietet sich f�r die Arbeit mit TeXnicCenter der Sumatra PDF-Viewer an. Dieser erm�glicht es, das PDF-Dokument w�hrend der Bearbeitung mit TeXnicCenter ge�ffnet zu lassen. Das PDF-Dokument aktuallisiert sich au�erdem nach der Ausgabe automatisch und die Ansicht bleibt an der entsprechenden Stelle im Dokument.\\

Der Sumatra PDF Viewer ist Freeware und sehr schlank.
Er liegt dieser Vorlage als "SumatraPDF-2.1.1-install.exe" bei, oder steht
auf der Seite des Herstellers zum Download bereit:\\

http://blog.kowalczyk.info/software/sumatrapdf/download-free-pdf-viewer-de.html \\

Nach der Installation muss noch die Einstellung des TeXnixCenter f�r die Verwendung von Sumatra angepasst werden.

Eine Anleitung findet sich in "Sumatra Einrichtung.pdf" oder hier:\\

http://www.texniccenter.org/resources/tutorials

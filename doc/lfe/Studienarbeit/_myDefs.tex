\usepackage[
%headinclude, 
%footinclude,
%headsepline,
%footsepline,
%plainfootsepline, 
%plainheadsepline,
automark
]{scrpage2}



\pagestyle{scrheadings}
\clearscrplain
\clearscrheadings
\ofoot[\pagemark]{\pagemark} %Seitenzahlen rechts unten, bitte!
\addtolength{\footskip}{-1cm} %Seitenzahlen ein bisschen h�her setzen


\newcommand{\bild}[4]{
  \begin{figure}[!hbt]
    \begin{center}
    %\centering
      %\vspace{1ex}
      \includegraphics[#2]{img/#1}
      \caption[#4]{\label{img.#1} #3}
    \end{center}
    %\vspace{1ex}
  \end{figure}
 }
 
%bild mit rahmen 
\newcommand{\rbild}[4]{
  \begin{figure}[!hbt]
    \begin{center}
    \fbox{
    %\centering
      %\vspace{1ex}
      \includegraphics[#2]{img/#1}
      }
      \caption[#4]{\label{img.#1} #3}
    \end{center}
    %\vspace{1ex}
  \end{figure}
 }
 
 

\newcommand{\fbild}[5]{
  \begin{floatingfigure}[r]{#5}
    \begin{center}
    %\centering
      %\vspace{1ex}
      \includegraphics[#2]{img/#1}
      \caption[#4]{\label{img.#1} #3}
    \end{center}
    %\vspace{1ex}
  \end{floatingfigure}
 }


\newcommand{\mytab}[4]{
    \begin{table}[!hbt]
    \begin{center}
      \vspace{1ex}
		\begin{tabular}{#2}
			\hline %hline here beacuse hline as first command in included \input-files produces errors
			    \input{tab/#1}
			\hline	    		
		\end{tabular}
          \caption[#4]{\label{tab.#1} #3}
    \end{center}
    \vspace{1ex}
\end{table}
}

\newcommand{\tausend}{\hspace {.5ex}} %50\t 000 => 50 000 tausender trenner 

\newcommand{\comment}[1]{} 

\newcommand{\mytabsmall}[4]{
    \begin{table}[!hbt]
    \begin{center}
      \vspace{1ex}
      {\tiny
		\begin{tabular}{#2}
			\hline %hline here beacuse hline as first command in included \input-files produces errors
			    \input{tab/#1}		
			\hline
		\end{tabular}
		}
          \caption[#4]{\label{tab.#1} #3}
    \end{center}
    \vspace{1ex}
\end{table}
}


% foldersymbol
\newcommand{\folder}[1]{
      \hspace{-3ex}
 \scalebox{0.40}[0.40]
{\includegraphics[viewport=14 15 48 48,clip]{img/folder.pdf}  }
 \bf #1 \normalfont \tiny
 }

% dokumentsymbol 
\newcommand{\dokument}[1]{
      \hspace{-3ex}
 \scalebox{0.60}[0.50]
{\includegraphics[viewport= 22 16 40 42, clip]{img/dokument.pdf}  }
 \bf #1 \normalfont \tiny
 }

% ANWENDUNGSBEISPIELE--------------------------------------------------------------------
%    \bild{meinbild.pdf}{viewport=0 0 200 350, clip=true}{BILDUNTERSCHRIFT}{Eintrag fuers Abbildungsverzeichniss}
%    \bildscale{meinbild.pdf}{viewport=0 0 200 350, clip=true}{BILDUNTERSCHRIFT}{Eintrag fuers 
% 																																									Abbildungsverzeichniss}{scalex}{scaley}
%    \mytab{datei_in_der_die_tabelle_ist.tex}{lrrr <= Format }{TABELLENUNTERSCHRIFT}{Eintrag fuers Tabellenverzeichnis}
%    \dokument[datei$.$mat] % punkt maskieren '$.$' wegen 'dirtree'-umgebung
%    \folder[mydirectory]



% macht den rand schoener (optischer randausgleich)
\usepackage[activate=normal]{pdfcprot}


\usepackage{units}
%\unit[Wert]{Einheit}
%\unitfrac[Wert]{Z�hler}{Nenner}

\newcommand{\myunit}[1]{\,\unit{#1}}  % ohne []
\newcommand{\myunitrm}[1]{\,\mathrm{\unit{#1}}} % mit []
\newcommand{\myunitfrac}[2]{\,\mathrm{\unitfrac{#1}{#2}}}
%
%U = 2.459\, 123 \myunit{V}
%\sigma = \frac{F}{A} = 133.432\, 192\, 233 \myunitfrac{N}{mm^{2}}

\usepackage{url}

\usepackage[plainpages=false,pdfpagelabels=true,bookmarks=true, 
						%bookmarksopen=true, % waehrend testen offen lassen spaeter zumachen
						pdfpagemode={UseOutlines},pdfstartview={FitV},pdfborder={0 0 0},
						pdfauthor={\myname}, 
						pdftitle={\mythema},
						%pdfsubject={\mythema}, % optional
						%pdfkeywords={Schluesselwort1,Schluesselwort2,Schluesselwort3},
						colorlinks=true,
						linkcolor=black,
						filecolor=black,
						urlcolor=black,
						citecolor=black,
						hypertexnames=true,
						pdfpagelabels=true,
						hyperindex=true,
						linktocpage=true,
						pagebackref=true
						]{hyperref}
						
						

	\usepackage[round]{natbib}
    \usepackage{amsmath} 
    \usepackage{dirtree}
    \usepackage{algorithm}
    \usepackage[noend]{algpseudocode}
    \usepackage{subcaption}
	\usepackage{geometry}
%	\geometry{left=3.5cm,textwidth=15cm,top=2.5cm,textheight=24.5cm}
%	\geometry{left=2.5cm, right=2.5cm, bottom=3cm, top=2.5cm}
	\geometry{left=3cm, right=2.6cm, bottom=3cm, top=2.5cm}
	
	\usepackage{makeidx}
	\makeindex

\usepackage{setspace}
%\onehalfspacing %Zu klein im Vergleich zu Word
\setstretch{1.54} %Simuliert den 1,5 fachen Zeilenabstand von Word (Bei dieser Schriftart und dieser Schriftgr��e)
\renewcommand{\arraystretch}{1.54} %Zeilenabstand in Tabellen! 

\setlength{\parindent}{0pt} %Kein Einr�cken der ersten Zeile eines Absatzes
\setlength{\headheight}{1.1\baselineskip}

\usepackage{titlesec} %Dieser Abschnitt definiert die �berschriften
\titleformat{\chapter}[block]{\large\bfseries}{\thechapter}{.5em}{}
\titleformat{\section}[block]{\large\bfseries}{\thesection}{.5em}{}
\titleformat{\subsection}[block]{\large\bfseries}{\thesubsection}{.5em}{}
\titleformat{\subsubsection}[block]{\large\bfseries}{\thesubsubsection}{.5em}{}

%
%
%\setlength{\headsep}{1cm}
%\setkomafont{pagefoot}{\normalfont} %Sonst wird Fu�zeile kursiv
%\setkomafont{pageheadfoot}{\normalfont} %Sonst wird Kopfzeile kursiv
%
%%\pagestyle{scrheadings}
\clearscrheadfoot % clear header and footer
%\automark[section]{chapter}
%\ihead[]{\leftmark} % header left part
%%\ohead[]{\rightmark} % header right part
%\ohead[]{\scalebox{0.05}{\includegraphics{img/lfe.pdf}}} % header right part
%\ifoot{\small \myarbeit \space \myname} % footer left part  
%%\cfoot{\vspace{-0.5cm}\mythema} % footer middle part
%%\ofoot{Seite\ {\pnumfont \pagemark}\ von {\pnumfont \pageref{LastPage}}} % footer right part
\ofoot{\small {\pnumfont \pagemark}} % footer right part
%\setheadsepline{0.3pt} % set header seperate line
%\setfootsepline{0.3pt} % set footer seperate line
%%\setheadwidth{textwithmarginpar}
%%\setfootwidth{textwithmarginpar}



\usepackage[dvips]{graphicx}
\usepackage{floatflt,epsfig} 

% \usepackage{ngerman} % neue deutsche Rechtschreibung
% \usepackage[latin1]{inputenc} % latin1-Kodierung f�r Umlaute
% \usepackage[ngerman]{babel}  % Silbentrennung
\usepackage[T1]{fontenc}
\usepackage[scaled]{uarial} % Schriftart Arial
\usepackage[font=small,labelfont=it]{caption} %�berschriften kursiv


\renewcommand*\familydefault{\sfdefault} %Sonst werden die Header nicht in der entsprechenden Schriftart dargestellt
\renewcommand\thefigure{\arabic{chapter}-\arabic{figure}} %Nummerierung der Abbildungen mit - statt .
\renewcommand\thetable{\arabic{chapter}-\arabic{table}} %Nummerierung der Tabellen mit - statt .

\setcounter{secnumdepth}{3} %Tiefe der Numerierung
\setcounter{tocdepth}{3} %Tiefe des Inhaltsverzeichnisses
\usepackage{graphicx}

\clubpenalty = 10000 % Schusterjungen vermeiden
\widowpenalty = 10000 % Hurenkinder vermeiden
\displaywidowpenalty = 10000 % und nochmal f�r Formeln

\usepackage{graphicx} % Bilder
\usepackage{color} % Farben
\usepackage{colortbl} % tabellen einf�rben
\usepackage{floatflt} % graphiken mit textumfluss
% \usepackage{subfigure} % graphiken nebeneinander mit (a) (b)
\usepackage[absolute]{textpos} % absolute positioning


\usepackage{scrhack}
\usepackage{listings} % programmcode als listings darstellen

%workaround
% \addto\captionsngerman{
%  \renewcommand{\figurename}{Abbildung}%
% }

